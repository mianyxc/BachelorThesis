%本章探讨改进失效现象

\chapter{改进失效的风险}

在前一章的证明过程中,我们通过添加项的方式对不等式进行放缩,从而证明了$\sum_{i=1}^N\xi_i > \xi$是严格成立的。但是这一推导过程中,我们默认了$z_\alpha>0$。由$z_\alpha$的定义可知
\[
z_\alpha>0 \Longleftrightarrow \eta=1-\alpha > 0.5
\]
也就是说,只有当企业设定的服务水平高于50\%时,才能够通过保留在制品库存降低企业库存总量。

当$\eta=1-\alpha=0.5$时,有$z_\alpha=0$,参考公式\ref{eq:改进前后库存比较i}的推导过程,可知此时$\sum_{i=1}^N\xi_i - \xi = 0$,即改进前后企业的总库存没有发生变化。

当$\eta=1-\alpha<0.5$时,有$z_\alpha<0$,参考公式\ref{eq:改进前后库存比较i}的推导过程,可以发现此时的不等式放缩方向是相反的,应该得到$\sum_{i=1}^N\xi_i - \xi < 0$,即改进后反而使得企业的总库存增加。

由此可见,当需求服从正态分布时,在某些服务水平下,将成品库存合并前移到在制品库存不仅没有起到改进效果,反而会使总库存增大。我们将这种情况称为改进失效。

本章将结合常用的需求分布函数,详细讨论改进失效现象与需求分布、服务水平等因素的关系。







\section{右偏斜的需求分布一定存在改进失效的可能}

(由于...,我们平常用到的需求分布都是右偏斜的)

(此处补充一些文献提到右偏斜的需求分布对substitution的影响)

假设有$n$种颜色的成品,其需求$D_i(i=1,2,\ldots,n)$独立同分布且该分布是右偏斜的,累积分布函数都为$F$,均值为$\mu$,方差为$\sigma^2$。将成品库存合并前移到在制品库存后,在制品的总需求为$D_n=\sum_{i=1}^nD_i$,其累积分布函数为$F_n$,均值为$n\mu$,方差为$n\sigma^2$。

设每种颜色的成品库存量为$s$,则对应的服务水平为$\eta=F(s)$。由$F_n$的定义知
\begin{equation}
F_n(ns) = P(D_n<ns) = P\left(\frac{\sum_{i=1}^nD_i}{n}<s\right)
\label{eq:Fn转为均值形式}
\end{equation}

根据中心极限定理,当$n\to\infty$时,$\frac{1}{n}\sum_{i=1}^nD_i$的分布趋近于正态分布$N(\mu,\sigma^2/n)$。设正态分布$N(\mu,\sigma^2/n)$的累积分布函数为$G$。由中心极限定理得
\begin{equation}
\lim_{n\to\infty}P\left(\frac{\sum_{i=1}^nD_i}{n}<s\right) = \lim_{n\to\infty}\Phi\left(\frac{s-\mu}{\sigma/\sqrt{n}}\right) = \lim_{n\to\infty}G(s)
\label{eq:中心极限定理}
\end{equation}
其中$\Phi$为标准正态分布的累积分布函数。由公式\ref{eq:Fn转为均值形式}和\ref{eq:中心极限定理}可知
\begin{equation}
\lim_{n\to\infty}[F_n(ns)-G(s)]=0
\label{eq:Fn与G的极限形式}
\end{equation}

接下来我们将证明,对任何的右偏斜需求分布,一定存在一个区间,当服务水平在此区间内时,就存在改进失效的可能。已知需求$D_i$的分布是右偏的,因此有
\begin{equation}
F(\mu) > 0.5 = \Phi(0) = G(\mu)
\label{eq:右偏斜的性质}
\end{equation}
由$F$和$G$的连续性,至少存在一个区间$[\mu,s_0)$满足
\[
F(s) > G(s),\qquad \forall s\in[\mu,s_0)
\]
令$\delta=F(s)-G(s)>0$。在公式\ref{eq:Fn与G的极限形式}中,根据极限的定义,存在一个正数$N$,使得对任意的$n>N$,都有
\begin{equation}
|F_n(ns)-G(s)| < \delta = F(s) - G(s)
\label{eq:根据极限的定义}
\end{equation}
公式\ref{eq:根据极限的定义}显示,对于任意的$s\in[\mu,s_0)$,都存在一个$N$,使得$n>N$时恒有$F_n(ns)<F(s)$。

我们知道$F(s)$代表需求量小于$s$的概率,即库存量$s$对应的服务水平。反过来,某个服务水平$\eta$也对应着一个库存量,我们将这个对应关系定义为函数$F^{-1}$。函数$F^{-1}$的表达式为
\[
F^{-1}(\eta) = \inf\left\{s\middle|F(s)\geq\eta\right\}
\]
其中$\inf$表示集合的下确界。同理可定义$F_n^{-1}$。

若企业需要满足的服务水平为$\eta$,则改进前和改进后的库存分别为$F^{-1}(\eta)$和$F_n^{-1}(\eta)$。若改进后的在制品库存大于改进前的各颜色成品库存之和,就说明改进失效了。也就是说,改进失效的具体表现是$F_n^{-1}(\eta)>nF^{-1}(\eta)$。下面我们将证明,当$F^{-1}(\eta)\in[\mu,s_0)$时,存在改进失效的可能。

现在假设企业的设定的服务水平为$\eta$,满足$\eta\in[F(\mu,F(s_0))$,则改进前各颜色成品的库存为$s=F^{-1}(\eta)\in[\mu,s_0)$。前面已经证明,对于任意的$s\in[\mu,s_0)$,都存在一个$N$,使得$n>N$时恒有
\begin{equation}
F_n(ns)<F(s)=\eta
\label{eq:改进前后服务水平的关系}
\end{equation}
根据反函数的性质,由于累积分布函数$F$和$F_n$是单调递增的,所以$F^{-1}$和$F_n^{-1}$也是单调递增的。公式\ref{eq:改进前后服务水平的关系}显示$F_n(ns)<\eta$,结合$F_n^{-1}$的单调性可知
\begin{equation}
F_n^{-1}(\eta) > ns
\label{eq:反函数单调性}
\end{equation}
把$s=F^{-1}(\eta)$代入公式\ref{eq:反函数单调性}有
\[
F_n^{-1}(\eta)>nF^{-1}(\eta)
\]
前面已经提到,上式即为改进失效的具体表现。

通过本节的分析,我们证明了,当各颜色成品需求服从独立同分布的任意一种右偏斜分布时,存在一个大于0.5的服务水平区间,当企业的服务水平位于这个区间时,存在改进失效的可能。








\section{任何服务水平都存在改进失效的可能}






















