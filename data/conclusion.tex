\chapter{总结与展望}





\section{总结}

本文主要从四个不同的方面探究了“合并同型号成品库存到在制品库存”这一改进方案可能产生的效果。本文在探究过程中采用的主要方式是数学公式的推导,这使得本文的结论具有一定的局限性——基于很多理想化的假设条件;但同时也赋予了这些结论更强的普适性——没有局限于某个具体的企业生产线,而是给出普遍意义的结论。

在研究改进方案是否能够起到改进效果时,本文从数学性质良好的正态分布入手,证明了改进方案的优化潜力。还通过数值实验,研究了系统中一些参数对改进效果的影响。结果表明,需求波动越大、服务水平越高的企业,在改进中获得的收益越大。

在研究改进失效现象时,本文既分析了数学性质良好的对称稳定分布,又分析了实际经常使用的右偏斜分布。同时也关注了服务水平的影响,探讨了两种不同的服务水平定义带来的不同结果。最后通过数值实验,找出参数影响改进失效现象的规律,提出了避免改进失效的方法,并且验证了之前证明的一些结论。

在研究需求相关性对改进效果的影响时,本文关注了很多文献都不曾注意的细节,把证明的基础建立在严密的构造上。讨论了极端状态下的改进效果,找到了需求相关性对改进效果的影响规律。得到结论后,又通过一些实际的数据为例,展示了企业应该如何利用这些结论来做决定。

研究延迟订单对改进效果的影响时,本文不拘泥于前面三部分的研究方法,而是采用了运筹学的一些方法和结论,得到了定性和定量的结果。同时,还给出了估算改进效果上限的方法,为企业的决策提供一个参考。

在这四部分的研究中,本文始终坚持扎实的作风,尽量将数学上的细节考虑周全,使用较为严谨的数学语言来进行推导,而不是通过自然语言进行叙述性的证明。因此,在限定的假设条件下,本文给出的结论都是可靠的,很少受到随机因素的干扰。







\section{展望}

本文对需求分布、服务水平、生产时间等参数都有一定的研究,但尚未进展到对提前期变化的研究。企业将成品的库存合并前移到在制品,那么从在制品到成品所需的生产时间势必对改进效果产生相当重要的影响。再加上提前期的波动,整个系统在时间方面可能受到的影响还很多。这一部分的研究也是比较有难度的。对时间问题的讨论可能需要继续使用马尔科夫过程,必要的时候还可能需要用到杰克逊网络。

除此之外,本文也没有涉及库存策略的影响。实际生产中对库存的控制往往是通过各种既定的策略来完成的,比如(s,S)策略,推动式生产或拉动式生产等等。当库存遵循这些策略的时候,改进方案还能起到效果吗?为了适应改进方案,库存策略中的一些参数应该取何值?

以上这些问题都是具有较强实际意义的,同时,它们也具有比较高的难度。若时间和精力允许,我将在未来的研究中尝试解决它们。





























