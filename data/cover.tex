
%%% Local Variables:
%%% mode: latex
%%% TeX-master: t
%%% End:
\secretlevel{绝密} \secretyear{2100}

\ctitle{在制品库存与产品库存的关系研究}
% 根据自己的情况选,不用这样复杂
\makeatletter
\ifthu@bachelor\relax\else
  \ifthu@doctor
    \cdegree{工学博士}
  \else
    \ifthu@master
      \cdegree{工学硕士}
    \fi
  \fi
\fi
\makeatother


\cdepartment[工业工程系]{工业工程系}
\cmajor{工业工程}
\cauthor{徐驰} 
\csupervisor{吴甦教授}
% 如果没有副指导老师或者联合指导老师,把下面两行相应的删除即可。
%\cassosupervisor{陈文光教授}
%\ccosupervisor{某某某教授}
% 日期自动生成,如果你要自己写就改这个cdate
%\cdate{\CJKdigits{\the\year}年\CJKnumber{\the\month}月}

% 博士后部分
% \cfirstdiscipline{计算机科学与技术}
% \cseconddiscipline{系统结构}
% \postdoctordate{2009年7月——2011年7月}

\etitle{The Impact of Reserving In-process Inventory on Safety Stock} 
% \edegree{Doctor of Science} 
\edegree{Bachelor of Engineering} 
\emajor{Industrial Engineering} 
\eauthor{Xu Chi} 
\esupervisor{Professor Wu Su} 
%\eassosupervisor{Chen Wenguang} 
% 这个日期也会自动生成,你要改么?
% \edate{December, 2005}

% 定义中英文摘要和关键字
\begin{cabstract}

随着制造业的发展,企业越来越重视库存的优化。对于某些从相同的半成品分化出来的产品,企业希望能够使其库存停留在未分化的在制品阶段,待收到订单之后再按要求加工,以此来减少该系列产品的总库存。本文主要研究这种库存改进策略的效果,分析需求分布、需求相关性、服务水平、生产能力等参数对改进效果的影响,为企业提供决策支持。


  
  
  
\end{cabstract}

\ckeywords{在制品库存, 安全库存, 风险分担, 延迟差异化}





\begin{eabstract} 

As manufacturing industry develops, enterprises are paying more attention to optimizing their inventory. There are some products that come frome the same kind of semi-product. Enterprise would like to keep these products at in-process inventory until specific orders arrive, so as to reduce the total inventory. This paper studies the sequence of the strategy, shows how the parameters such as the the distribution of demand, correlation of demand, service level, production capacity would affect the improvement of the strategy. These conclusions could be enterprises' decision support.





\end{eabstract}

\ekeywords{In-process inventory, Safety stock, Risk-pooling, Postponement}
