%本章讨论提前期对改进效果的影响

\chapter{提前期对改进效果的影响}

\section{需求正态分布的情况}

仍然以汽车保险杠生产为例。假设生产一个注塑件耗时$\tau_1$,喷涂一个注塑件耗时$\tau_2$。客户允许的备货时间是$\tau$。

$\tau_1$、$\tau_2$、$\tau$三者的大小关系发生变化的时候,方案的改进效果也会发生变化。

(此处需要分类讨论很多条)



\section{需求泊松到达的情况}

设需求到达的速率为$\lambda_1$、$\lambda_2$,注塑速率$\mu_1$,喷涂速率$\mu_2$。建模成马尔科夫(生灭问题),可以求稳态概率,State 0 的概率即为缺货概率。

需要证明两个独立的泊松流合并后仍然是泊松流,且到达速率为$\lambda_1+\lambda_2$。

还需证明两个输入输出均为泊松过程的服务台,可以合并视为一个泊松过程的服务台(参考Jackson网络相关章节)。





\section{(s,S)库存策略下的情况}

由于换模等因素的影响,企业的实际生产中不可能随时对某种产品补货。往往采用的是(s,S)策略,即库存降到s后开始生产该产品,一直补货到库存为S为止。这种策略下也能建模成马尔科夫模型。



