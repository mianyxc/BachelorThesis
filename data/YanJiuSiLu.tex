\chapter{研究思路}

\section{证明优化潜力存在性}

将成品库存合并前移到在途库存,会造成安全库存的变动。影响这个变动的因素是多方面的:风险分担(risk pooling)使安全库存减少;但在发货之前必须先把在途库存制成成品,相当于增加了提前期,使安全库存增大。

为了证明生产线确实存在优化潜力,我们从极端情况入手,先假设喷涂工序不消耗时间,即忽略库存前移对提前期的影响。如果这种情况下安全库存减少了,那么至少可以认为:新的库存点与成品阶段之间的工序耗时不多的情况下,库存合并前移能减少安全库存。如此即证明了生产线确实存在可优化的潜力。

研究这个问题时,可以首先从最简单的情况入手:假设某注塑件在喷涂阶段分化成两种不同颜色的保险杠成品,并且两种保险杠成品的需求服从独立同分布的正态分布。我们需要证明的是,在这种情况下,如果不保留两种保险杠成品的库存,而是保留喷涂前的注塑件在制品库存,是否能在保持原有服务水平的同时,使得安全库存总量降低。

在此基础上,可以把结论推广到两种不同颜色的保险杠需求独立,但服从不同的正态分布的情况。最后可以推广到多种不同颜色的保险杠需求独立,且服从不同的正态分布的情况。这样,我们就能够证明,当保险杠需求服从独立的正态分布时,将在制品库存保留在分化前的阶段,能够使安全库存总量降低。

最后,可以通过一些数值实验,探讨系统中的各参数对改进效果的影响。




\section{改进失效现象的研究}

通过前期的一些数值实验,我们发现一些值得注意的事情:在某些情况下,改进方案不仅不能使总库存减小,反而会使总库存变大。

这个现象的成因是什么?具体表现是什么?影响它的因素有哪些?这些问题都是值得探索研究的。

我们需要从不同的需求分布、不同的服务水平入手,讨论各种参数对这种现象的影响。只有我们找到引发这种现象的参数规律之后,才能总结出我们最希望得到的结论:如何判断企业当前环境下是否会发生改进失效,以及如何避免改进失效。






\section{需求相关性的影响}

将不同颜色相同型号的成品库存合并前移,成品之间的相关性必然会影响到方案的效果。因此需要分类讨论成品需求相互独立、需求正相关、需求负相关等情况。
\begin{description}
\item[独立]
成品需求相互独立,意味着一种颜色的保险杠需求量对相同型号另一种颜色的保险杠需求量没有影响。这是最简单的假设,但有时候可能不符合实际情况。
\item[正相关]
某种颜色的保险杠需求变化,可能是由于该车型的市场需求变化。此时该车型的各种颜色需求可能同时变化。也就是说,相同型号不同颜色的保险杠需求可能倾向于同时增大或减小。这是生产实际中可能发生的情况。
\item[负相关]
如果某种车型的市场需求相对稳定,则意味着,一种颜色的需求增长会抑制另一种颜色的需求,表现为相同型号不同颜色的保险杠需求负相关。这也是生产实际中可能发生的情况。
\end{description}

不同的需求相关性会影响库存改进方案的效果。显然,成品需求负相关的情况下,此改进方案的效果将会更加突出。分析成品需求完全正相关和完全负相关的极端情况,能帮助我们明确改进方案的绩效范围,找到改进效果的上下界,从而更好地作出决策。

研究的最后,需要对企业生产线的实际数据进行统计分析,找出不同颜色相同型号的成品需求之间的实际相关性,并针对该结果做具体分析。



\section{延迟订单的影响}

在前面的思路中,我们始终默认每一期的需求对后一期没有影响,也就是说未满足的订单将会被直接舍弃。实际上,作为供应商,如果不能按时交货,有时需要将订单记下,尽快补货,并且承受一定的惩罚。这种订单称为延迟订单。在这一部分中,我们将会讨论有延迟订单的情况下,改进的具体效果。

由于前期的需求和库存会对后期造成一定的影响,此处需要引入马尔科夫过程来进行建模。特别的,如果需求和生产遵循泊松过程的话,我们可以利用一些已有的模型来进行研究,如生灭过程。以一些现有的数学结论作为基础,就能尝试一些定量的研究,如计算改进效果的上限等。








