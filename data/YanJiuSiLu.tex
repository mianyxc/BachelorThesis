\chapter{研究思路}

\section{证明优化潜力存在性}

将成品库存合并前移到在途库存,会造成安全库存的变动。影响这个变动的因素是多方面的:风险分担(risk pooling)使安全库存减少;但在发货之前必须先把在途库存制成成品,相当于增加了提前期,使安全库存增大。

为了证明生产线确实存在优化潜力,我们从极端情况入手,先假设喷涂工序不消耗时间,即忽略库存前移对提前期的影响。如果这种情况下安全库存减少了,那么至少可以认为:新的库存点与成品阶段之间的工序耗时不多的情况下,库存合并前移能减少安全库存。如此即证明了生产线确实存在可优化的潜力。

研究这个问题时,可以首先从最简单的情况入手:假设某注塑件在喷涂阶段分化成两种不同颜色的保险杠成品,并且两种保险杠成品的需求服从独立同分布的正态分布。我们需要证明的是,在这种情况下,如果不保留两种保险杠成品的库存,而是保留喷涂前的注塑件在制品库存,是否能在保持原有服务水平的同时,使得安全库存总量降低。

在此基础上,可以把结论推广到两种不同颜色的保险杠需求独立,但服从不同的正态分布的情况。最后可以推广到多种不同颜色的保险杠需求独立,且服从不同的正态分布的情况。这样,我们就能够证明,当保险杠需求服从独立的正态分布时,将在制品库存保留在分化前的阶段,能够使安全库存总量降低。

正态分布在实际生活中是普遍存在的,并且它拥有很多良好的数学性质,因此我们常常通过对正态分布的研究来作为问题的切入点。但是在讨论生产方面的问题时,我们也常常用泊松过程来模拟一些过程,比如服务的时间、设备的失效或者需求的到达。如果需求的到达是一个泊松过程,则一段时间内的需求就是服从泊松分布的。

可以证明,当泊松分布的参数$\lambda$较大时,可将泊松分布近似处理为正态分布,这样就能把问题转化为刚刚已解决的正态分布的情况。然而在企业的实际生产中,有一些产品的需求量不大,不适合正态近似,因此对泊松分布本身的研究也是必要的。由于泊松分布并不具有正态分布一样良好的数学性质,在对其进行纯数学上的证明遇到困难时,可以考虑用数值模拟的方式进行一些简单的实验。







\section{成品需求相关性的影响}

将不同颜色相同型号的成品库存合并前移,成品之间的相关性必然会影响到方案的效果。因此需要分类讨论成品需求相互独立、需求正相关、需求负相关等情况。
\begin{description}
\item[独立]
成品需求相互独立,意味着一种颜色的保险杠需求量对相同型号另一种颜色的保险杠需求量没有影响。这是最简单的假设,但有时候可能不符合实际情况。
\item[正相关]
某种颜色的保险杠需求变化,可能是由于该车型的市场需求变化。此时该车型的各种颜色需求可能同时变化。也就是说,相同型号不同颜色的保险杠需求可能倾向于同时增大或减小。这是生产实际中可能发生的情况。
\item[负相关]
如果某种车型的市场需求相对稳定,则意味着,一种颜色的需求增长会抑制另一种颜色的需求,表现为相同型号不同颜色的保险杠需求负相关。这也是生产实际中可能发生的情况。
\end{description}

不同的需求相关性会影响库存改进方案的效果。显然,成品需求负相关的情况下,此改进方案的效果将会更加突出。分析成品需求完全正相关和完全负相关的极端情况,能帮助我们明确改进方案的绩效范围,找到改进效果的上下界,从而更好地作出决策。

研究的最后,需要对企业生产线的实际数据进行统计分析,找出不同颜色相同型号的成品需求之间的实际相关性,并针对该结果做具体分析。









\section{提前期变化的影响}

在实际生产中,库存前移会导致供货提前期发生变化。以汽车保险杠的生产为例,假设原定的供货提前期是$T$,喷涂工序所需时间是$t$,在新的库存方案下,由于需要先喷涂再发货,相当于把留给企业的反应时间缩减到了$T-t$,这就对企业的敏捷性和柔性提出了挑战。

显然,当$t$较小时,我们可以忽略它,把问题转化为上面讨论过的情况。但是当$t$值大到不能忽略时,必然会影响到方案的效果,最糟糕的情况下甚至可能使安全库存增大。因此,研究中需要考虑提前期变化对改进方案的影响,希望能找到一定条件下,方案能产生改进效果的提前期临界点,作为决策支持。

研究提前期的影响时仍然可以以正态分布和泊松分布来研究。

正态分布的情况下,可以考虑相同库存总量情况下,改进前后的缺货概率大小。对于改进前后的系统来说,提前期内能够供货的最大量都等于成品库存与提前期内产量之和。以课题背景中提到的汽车保险杠生产为例:
\begin{description}
\item[改进前]
持有各颜色的保险杠成品库存。一旦发现需求过高,成品库存无法满足需求,就需要立即开工生产,先注塑,再喷涂,争取在提前期内生产出足够的数量。
\item[改进后]
不保留成品库存。当需求到来时,从在制品库存中选取相应数量的塑件,经过喷涂后发货;如果发现塑件总量不足以满足需求,则同时启动注塑,以补充塑件在制品库存。
\end{description}
注塑、喷涂各自的生产时间和提前持有的库存总量都会影响两种方案的绩效。此处需要进行更为细致的讨论。

泊松分布的情况下,由于需求是泊松到达,我们可以把问题建模成连续时间的马尔科夫过程,通过求解稳态概率来比较两种方案的缺货概率。由于生产过程不止一个阶段,可能还需要用到Jackson网络。






\section{最优库存策略的探索(待定)}

考虑$(S,s)$库存策略下如何确定最优的$S$和$s$。也可以考虑如何是否只把一部分库存前移能达到最优。






\section{实际生产数据的应用}

在理论研究的结论中代入企业的实际生产数据,检验方案的实际改进效果,为企业提供决策支持。