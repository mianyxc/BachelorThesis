\chapter{引言}

\section{课题背景}

随着MIS、MRP等技术在汽车制造业的应用,汽车行业对零部件供应的及时性和服务水平要求越来越高。

G公司是全球最大的汽车制造商之一。M公司作为汽车零部件供应商,为G公司某些型号的汽车生产保险杠。

保险杠的生产流程需要经过注塑、喷涂、装配三个阶段,下面对每个阶段的作用和特点作简要介绍。

\begin{description}
\item[注塑] 注塑是保险杠生产的第一步,即通过注塑机器和模具,制造出特定形状的保险杠。注塑不同型号的保险杠时需要换模,换模时间相当长。为了提高生产效率,需要在注塑阶段设置合适的生产批量,尽量做到少换模。公司当前拥有注塑机器3台,生产的保险杠型号共十余种,并且要求每台机器每天换模不超过2次。这种策略下,每天能投入生产的保险杠型号是有限的,因此有必要保持一定量的安全库存。

\item[喷涂] 喷涂的作用是对注塑成型的保险杠上色。每种型号的保险杠有几种不同的颜色可选。喷涂工序是在连续运转的环形传送带上完成的,工人在某个位置将未喷涂的保险杠注塑件挂到传送带上的支架上,传送一段距离后进入喷涂机器,完成喷涂后在某个位置取下保险杠,然后空支架又运转到上货处。喷涂机器更换喷涂颜料需要一定的时间和人力,因此也要尽量设置合理的喷涂批量,减少颜料更换次数。

\item[装配] 装配是指给保险杠安装必要的配件。经过前两道工序,保险杠的主体已经基本完成。在打包发往G公司之前,还需要在保险杠上安装一些配件。这道工序耗时不多,且相当稳定,一般能保证在规定时间内完成装配并发货。

\end{description}


\section{问题描述}

如前所述,受限于注塑机器的换模次数,M公司无法在一天内生产每一种型号的保险杠塑件,因此需要保有足量的安全库存,以应对可能发生的需求波动。不仅如此,由于每种型号的保险杠有不同颜色之分,进一步增大了需求的不确定性。如果每种型号每种颜色的保险杠都保有足量安全库存,无疑使得安全库存量变得更大。

公司希望能在不影响服务水平、不降低生产效率的前提下,通过采用合理的生产和库存策略,减少安全库存总量。

\section{研究目标}

通过数学模型分析将成品安全库存合并前置到在制品库存是否有助于减少安全库存,并进一步分析此改进方案的适用范围和优化效果。最后根据前面的分析,给出一些定性或定量的结论,帮助企业对改进方案可能产生的效果建立初步的认识,以决定是否值得继续推进方案。


\section{意义和价值}

对于一般企业而言,在不降低服务水平和生产效率的基础上,通过库存合并前移减少安全库存,能够减少生产线的库存成本,节省场地和设备,解放更多的流动资金,对生产线的生产柔性有一定的提升。

但是,企业作出改进的同时也会付出一些必要的成本。因此,企业希望能在改进尚未开始的时候,就对改进方案可能带来的利益和后果有一定的了解。如果在改进之前发现改进不能起到正向的效果,或者改进的收益太小,企业就能避免无效的投入;如果在改进之前发现改进存在巨大的潜在收益,企业就可以投入更多的精力,充分挖掘改进的效果。
